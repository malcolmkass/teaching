\documentclass[compress]{beamer}
%\mode<presentation>
\usetheme{Madrid}
%\usetheme{Boadilla}
%\usetheme{AnnArbor}
%\usetheme{CambridgeUS}
\useoutertheme[footline=authortitle,subsection=false]{miniframes}
\usecolortheme{lily}
\setbeamerfont{normal text}{series=\mdseries}
\setbeamerfont{item}{series=\mdseries}
\setbeamerfont{itemize/enumerate body}{series=\mdseries}
\setbeamerfont{itemize/enumerate subbody}{series=\mdseries}
\usepackage{graphicx}
%\usepackage{caption}
%\usepackage{subcaption}
\usepackage{tikz}
\usepackage{pgfplots}
\pgfplotsset{compat=1.18}
\usepackage{booktabs}
\newcommand{\mykeywords}{Student article presentations}

%\usepackage{enumitem}% http://ctan.org/pkg/enumitem
\setbeamercolor{section in toc}{fg=dark_blue}
\setbeamercolor{itemize item}{fg=dark_blue}
\usepackage{color}
\usepackage{subcaption}
\usepackage{hyperref}
\usepackage{pdfpages}
\usepackage{appendixnumberbeamer}
\usepackage[english]{babel}
\usepackage{csquotes}
% \usepackage{blindtext} % only needed for filler text; safe to remove if unused
%\usepackage{beamerthemesplit} % Activate for custom appearance
%%%%%%%%%%%%%%%%%%%%%%%%%%%%%%%%%%%%%%%%%%%%%%%%%%
\definecolor{purple}{RGB}{128,100,162}
\definecolor{dk_purple}{RGB}{96,75,120}
\definecolor{maroon}{RGB}{149,55,53}
\definecolor{lt_maroon}{RGB}{175,101,65}
\definecolor{mute_blue}{RGB}{55,96,146}
\definecolor{mute_green}{RGB}{79,130,40}
\definecolor{dark_blue}{RGB}{32,64,97}
\definecolor{lt_green}{RGB}{195, 214, 155}
\definecolor{grey}{RGB}{125, 125, 125}
\definecolor{green}{RGB}{0, 182, 0}
\definecolor{blue}{RGB}{0, 0, 255}
\definecolor{teal}{RGB}{0, 105,128}
%%%%%%%%%%%%%%%%%%%%%%%%%%%%%%%%%%%%%%%%%%%%%%
\setbeamercolor{title}{fg=dark_blue}
\setbeamercolor{frametitle}{fg=dark_blue}
%\setbeamertemplate{footline}{}
\setbeamertemplate{footline}{%
  \leavevmode%
  \hbox{%
    \begin{beamercolorbox}[wd=.70\paperwidth,ht=2.5ex,dp=1ex,left]{title in head/foot}%
      \usebeamerfont{title in head/foot}%
      \mykeywords \hspace{1em}--\hspace{1em} \insertsubsectionhead
    \end{beamercolorbox}%
    \begin{beamercolorbox}[wd=.30\paperwidth,ht=2.5ex,dp=1ex,right]{date in head/foot}%
      \usebeamerfont{date in head/foot}%
      \insertshortdate{}\hspace*{2ex}%
      \insertframenumber{} / \inserttotalframenumber\hspace*{2ex}%
    \end{beamercolorbox}%
  }%
  \vskip0pt%
}
\setbeamerfont{footnote}{series=\mdseries}
\usepackage{caption}
\captionsetup[figure]{labelformat=empty}% redefines the caption setup of the figures environment in the beamer class.
\usefonttheme{professionalfonts}
\usepackage[style=authoryear, backend=bibtex]{biblatex}
\addbibresource{oratory.bib} 
\renewcommand*{\nameyeardelim}{\addcomma\addspace}
\let\oldparencite\parencite
\renewcommand{\parencite}[1]{{\mdseries\oldparencite{#1}}}

\title{Student Presentations of Economic Journal Articles}
\subtitle{Motivations, process, outcomes, lessons learned}
\author{
	Malcolm Kasf \& 
	John Soriano \\
	\vspace{0.5cm}
    {\it University of Dallas} \\
	\href{mailto:mkass@udallas.edu}{mkass@udallas.edu} \\
    \href{mailto:jsoriano@udallas.edu}{jsoriano@udallas.edu} }
    
\date{\today}


\begin{document}
	
	\begin{frame}
		\titlepage
	\end{frame}


\section{Goal and context}

\begin{frame}\frametitle{What we are trying to accomplish}
\begin{itemize}
  \item Build discipline-specific literacy: reading ``how economists argue'' in peer-reviewed work
  \item Build transferable speaking skills: clear claims, evidence, and explanation for a live audience
  \item Use student presentations to expand topic coverage without adding lecture time
\end{itemize}
\vspace{0.35cm}
\small
This talk uses materials from an ECO 4338 Public Finance syllabus and a Public Finance paper list/module, with the university's ``Art of Oratory'' emphasis as a guiding frame.
\end{frame}

\begin{frame}\frametitle{Why journal-article presentations (in econ, specifically)}
\begin{itemize}
  \item Economics is argument + method: theory, identification, assumptions, welfare criteria
  \item Students need practice extracting ``the spine'' of a paper (what, why, how, so what)
  \item Peer teaching creates repetition with variation: many papers, many presenters, shared vocabulary
\end{itemize}
\end{frame}

\section{Motivations}
\subsection{Oratory}

\begin{frame}\frametitle{Institutional motivation: ``Art of Oratory''}
\begin{itemize}
  \item The presentation requirement is explicitly framed as part of the university's ``Art of Oratory'' goal
  \item The aim is not ``more talking''; it is structured argument: claim $\rightarrow$ model/evidence $\rightarrow$ implications
  \item Presentations also function as a public rehearsal of professional norms (clarity, citations, Q\&A)
\end{itemize}
\end{frame}

\subsection{Course design}

\begin{frame}\frametitle{Assignment at a glance (ECO 4338)}
\begin{itemize}
  \item Weight: Literature Review Presentations / Paper presentation = 8\% of course grade
  \item Core prompt: present motivation, theoretical setup, and implications/results
  \item Structure: students are organized into groups by theme (papers on a similar topic)
  \item Instructor support: ``I will help you in this endeavor'' (coaching + scaffolding)
\end{itemize}
\end{frame}

\begin{frame}\frametitle{What ``present the paper'' means (a usable template)}
\begin{enumerate}
  \item Question: what is the economic problem and why does it matter?
  \item Mechanism: model, key assumptions, or identification strategy
  \item Result: main empirical/theoretical finding (ideally 1 figure/table)
  \item Implications: welfare, policy, or ``what would change if we believe this?''
  \item Limits: what the paper does \emph{not} show; external validity / missing margins
\end{enumerate}
\end{frame}

\section{Process}
\subsection{Choosing papers}

\begin{frame}\frametitle{What students actually presented (Public Finance paper menu)}
\begin{itemize}
  \item Public choice: primer + JEP papers (via AEA links in the course module)
  \item Externalities: ``clean air'' capitalization-style paper; sugar externalities (JEP)
  \item Capital gains / unrealized gains: SSZZ, policy commentary, WSJ piece, supporting threads
  \item Market vs.\ government failure \& Coase: institutional comparisons; Coase theorem (JEL/JEP-style)
  \item Cost-benefit: overview + applied paper (e.g., dredging / project evaluation)
\end{itemize}
\end{frame}

\begin{frame}\frametitle{Concrete examples (selected links from the course module)}
\small
\begin{itemize}
  \item Public choice primer: \href{https://iea.org.uk/publications/research/public-choice-a-primer}{iea.org.uk/publications/research/public-choice-a-primer}
  \item JEP on money in politics: \href{https://pubs.aeaweb.org/doi/pdfplus/10.1257/089533003321164976}{10.1257/089533003321164976}
  \item Externalities (sugar): \href{https://pubs.aeaweb.org/doi/pdfplus/10.1257/jep.33.3.202}{10.1257/jep.33.3.202}
  \item Unrealized capital gains debate: \href{https://zidar.princeton.edu/sites/g/files/toruqf3371/files/documents/sszz.pdf}{zidar.princeton.edu/.../sszz.pdf}
  \item Coase theorem (JEL): \href{https://www.aeaweb.org/articles?id=10.1257/jel.20191060}{10.1257/jel.20191060}
  \item Cost-benefit overview (JEP): \href{https://pubs.aeaweb.org/doi/pdfplus/10.1257/jep.12.4.133}{10.1257/jep.12.4.133}
\end{itemize}
\end{frame}

\begin{frame}\frametitle{Scaffolding: students do not start from zero}
\begin{itemize}
  \item Curated links (paper lists + recommended primers) to reduce ``search costs''
  \item Shared slide resources: topic-specific decks for key concepts (externalities, Coase, cap gains, etc.)
  \item Group structure to pool skills (one student strong on theory, one on data, one on speaking)
  \item Instructor checkpoints to reduce surprises late in the semester
\end{itemize}
\end{frame}

\subsection{Timeline}

\begin{frame}\frametitle{Timeline and checkpoints (example from syllabus)}
\begin{itemize}
  \item Late October: ``Tell me about your paper''
  \item Early November: outline + sources for evaluation
  \item Draft milestone: rough drafts due (example date listed in schedule)
  \item Presentation window: scheduled paper presentations (example date listed in schedule)
\end{itemize}
\end{frame}

\section{Assessment}
\subsection{Rubric}

\begin{frame}\frametitle{How I grade: separate \textit{content} from \textit{delivery}}
\begin{itemize}
  \item Content (econ): accurate mechanism, correct interpretation, appropriate welfare language
  \item Evidence use: 1--2 central figures/tables; explain axes, magnitudes, and counterfactual
  \item Narrative: a defensible thesis and a clear ``so what''
  \item Delivery: pacing, slide design, and Q\&A handling
\end{itemize}
\vspace{0.25cm}
\small
The syllabus also includes an explicit paper rubric structure (e.g., thesis/intro, research quality, analysis, conclusion); those dimensions map cleanly into presentation grading.
\end{frame}

\begin{frame}\frametitle{A simple rubric faculty can reuse (one-slide version)}
\begin{tabular}{p{0.28\linewidth} p{0.68\linewidth}}
\toprule
Category & What ``excellent'' looks like \\
\midrule
Motivation & Real-world stakes + economic question stated crisply \\
Setup/ID & Correct assumptions or empirical strategy; no ``hand-waving'' \\
Result & One main result explained with magnitudes and units \\
Implications & Links result to policy/welfare tradeoffs (efficiency vs.\ equity) \\
Communication & Slides readable; presenter guides audience; handles questions \\
\bottomrule
\end{tabular}
\end{frame}

\section{Outcomes and lessons}
\subsection{Outcomes}

\begin{frame}\frametitle{Observed benefits (for students)}
\begin{itemize}
  \item Reading skill: students learn to identify the paper's ``spine'' quickly
  \item Economic reasoning: practice translating models into predictions and welfare claims
  \item Confidence: repeated exposure to public speaking in a coached setting
  \item Peer learning: students see multiple applications across topics (externalities, public choice, taxation)
\end{itemize}
\end{frame}

\subsection{Challenges}

\begin{frame}\frametitle{Common challenges (and why they happen)}
\begin{itemize}
  \item Cognitive load: journal writing is dense; novices lose the main thread
  \item Method anxiety: identification, assumptions, and causal language are easy to misstate
  \item Uneven group contributions: ``division of labor'' can hide weak understanding
  \item Time: presentations take class time; the course must budget for it
\end{itemize}
\end{frame}

\begin{frame}\frametitle{Mitigations that helped}
\begin{itemize}
  \item Template-driven prep: required ``question--mechanism--result--implications'' structure
  \item Checkpointing: early topic approval, outline + sources, draft milestone
  \item Anchor visuals: require at least one key figure/table and explain it well
  \item Light coaching: short meetings to correct misunderstandings before presenting
\end{itemize}
\end{frame}

\subsection{Transferability}

\begin{frame}\frametitle{How other disciplines can adapt this}
\begin{itemize}
  \item Start with curated readings (2--4 per topic) to reduce search frictions
  \item Provide a one-page presentation template (claims, evidence, limitations)
  \item Use structured peer feedback (two strengths + one question)
  \item Grade process + product: checkpoints count, not just the final talk
\end{itemize}
\end{frame}

\begin{frame}\frametitle{Closing}
\begin{itemize}
  \item Student article presentations can be a high-leverage teaching tool when scaffolded
  \item They align well with an oratory-focused institutional mission
  \item The main design problem is reducing cognitive load while keeping intellectual rigor
\end{itemize}
\vspace{0.5cm}
\centerline{\Large Questions?}
\end{frame}

\end{document}
